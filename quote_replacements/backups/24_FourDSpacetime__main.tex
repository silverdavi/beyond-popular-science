Three spatial dimensions plus one time dimension ($D=4$) satisfy multiple independent physical constraints that fail in other dimensionalities. Gravitational orbits destabilize, many familiar interactions become non-renormalizable for $D>4$ (and super-renormalizable for $D<4$), wave propagation develops trailing echoes, and atomic structures can fail to be stable when $D \ne 4$. The temporal dimension count is equally constrained. Multiple time dimensions destroy the well-posedness of the initial value problem for hyperbolic differential equations like the wave equation, rendering physics unpredictable. Additional time dimensions generically spoil causality and energy positivity.

In classical potential theory, the spatial decay of fields from a point source — or any spherically symmetric mass, regardless of its internal dimensionality — follows a general scaling law determined by Gauss’s theorem: the flux through a sphere in $n$ spatial dimensions scales with its surface area, yielding a radial dependence of $1/r^{n-1}$ for the field and $1/r^{n-2}$ for the associated potential. In $n=3$ this produces the inverse-square law that governs Newtonian gravity and electrostatics. This particular falloff enables stable bound orbits under central forces, since it balances centripetal acceleration with potential curvature. In $n>3$, the force falls too quickly to support closed, radially stable Kepler orbits; in $n<3$, the dynamics cease to admit Kepler-type closed, radially stable motion.

Wave propagation obeys Huygens' principle only in odd spatial dimensions $n\ge 3$. In $3+1$ spacetime, a localized disturbance generates a sharp spherical wavefront without trailing components. In even spatial dimensions, persistent field residuals remain after the main wave passes, blurring temporal boundaries between cause and effect as the Green’s function of the wave equation has tails inside the light cone.

Quantum field theory imposes stringent dimensional restrictions on interaction consistency. Renormalizability — the ability to absorb divergences into a finite set of physical parameters — depends on the dimensional scaling of coupling constants. In $D=4$, key interactions such as $\phi^4$ theory, quantum electrodynamics, and non-abelian gauge theories feature dimensionless couplings, rendering loop corrections manageable via renormalization group techniques. In $D>4$, the same interactions become non-renormalizable, requiring an infinite tower of counterterms. In $D<4$, they become super-renormalizable.

The manifold $\mathbb{R}^4$ exhibits an anomaly in differential topology: it admits uncountably many smooth structures that are pairwise non-diffeomorphic yet topologically equivalent. These exotic $\mathbb{R}^4$s violate the standard equivalence between smooth and topological manifolds. No analogous phenomenon occurs in dimensions $n \ne 4$. This breakdown is intertwined with deep four-dimensional phenomena revealed by gauge theory, notably Donaldson invariants and Seiberg–Witten theory; the smooth 4D Poincaré conjecture itself remains open.

In the algebraic classification of normed division algebras over $\mathbb{R}$, there exist only four: $\mathbb{R}$ (dimension 1), $\mathbb{C}$ (dimension 2), $\mathbb{H}$ (dimension 4), and $\mathbb{O}$ (dimension 8). Of these, the quaternions $\mathbb{H}$ preserve associativity while the octonions $\mathbb{O}$ do not. They form the algebraic underpinning of spinor representations and enable the group isomorphism $\mathrm{SU}(2) \cong \mathrm{Spin}(3)$, which double-covers the rotation group $\mathrm{SO}(3)$. This supports the representation theory of spin-$\tfrac{1}{2}$ particles and the construction of Dirac spinors. No higher-dimensional associative division algebra exists, and the non-associativity of octonions complicates their use in comparable representation frameworks, though they underlie exceptional Lie groups (e.g., $G_2$, $F_4$, $E_6$–$E_8$).

The necessity of spinors in four dimensions emerges from a tension between quantum mechanics and relativity. Schrödinger's equation is linear in time derivatives, while relativistic energy obeys $E^2 = p^2c^2 + m^2c^4$, quadratic in energy. Dirac sought to linearize the wave operator — to extract a "square root" of the d'Alembertian $\square = \partial_t^2 - c^2\nabla^2$. Just as $x^2 + y^2$ cannot be factored into $(ax+by)$ using real numbers, this operator resists scalar factorization. The solution requires anticommuting coefficients, matrices $\gamma^\mu$ satisfying $\{\gamma^\mu, \gamma^\nu\} = 2\eta^{\mu\nu}$. In four-dimensional spacetime, the minimal representation uses $4\times 4$ matrices, forcing the wavefunction to be a four-component spinor rather than a scalar. The four components do not represent spatial directions, but two particle states and two antiparticle states, each with two spin orientations. Antimatter is a result of this mathematical necessity from the requirement to linearize energy in $D=4$ spacetime. The restriction to three spatial dimensions is important: the rotation group $\mathrm{SO}(3)$ is unique in admitting a double cover $\mathrm{Spin}(3) \cong \mathrm{SU}(2)$ that links vector and spinor representations through this square-root relationship.

In general relativity, the uniqueness of black hole solutions — encapsulated by the no-hair theorems — holds in four-dimensional, asymptotically flat spacetime under suitable regularity and symmetry assumptions. Theorems by Israel, Carter, and Robinson prove that stationary black holes in $D=4$ are characterized entirely by mass, charge, and angular momentum. In higher dimensions, this rigidity fails. New solutions emerge with toroidal or ring-like horizons, including black rings and black strings, and the solution space displays richer phases.

The quantum mechanical stability of atomic matter depends sensitively on the spatial dimension $n$. For hydrogen-like atoms with a $1/r^{\,n-2}$ potential when $n\ge 3$, the familiar Coulombic spectrum arises for $n=3$. In $n>3$, the potential decays too rapidly to maintain binding; in $n=2$, the potential becomes logarithmic. Chemical bonding patterns also require three dimensions. Tetrahedral carbon and chiral centers depend on $\mathrm{SO}(3)$ symmetry. In $n=2$, bonding is planar and chirality is lost; in $n>3$, additional rotational degrees of freedom would alter biochemical recognition.

I recommend watching \href{http://youtu.be/u5DLpAqX4YA&t=1170s}{Mikhail Gromov's lecture on the topic} (minute 19:30 in the video titled “What is a Manifold? - Mikhail Gromov”) where Gromov traces the exceptional nature of four dimensions to the arithmetic identity $4 = 2 + 2$. A four-element set partitions into two pairs in exactly three ways: $\{\{1,2\}, \{3,4\}\}$, $\{\{1,3\}, \{2,4\}\}$, and $\{\{1,4\}, \{2,3\}\}$. The number of such partitions for a set of size $2n$ is $(2n-1)!! = (2n-1)(2n-3)\cdots 3 \cdot 1$. For $n=2$: three partitions. For $n=3$: fifteen partitions. For $n=4$: one hundred and five partitions. Only when $n=2$ does the partition count (3) remain smaller than the set size (4), enabling the symmetric group $S_4$ to map onto the smaller group $S_3$.

For odd-sized sets, no symmetric pair partitions exist. Only the four-element set achieves both symmetry (all parts equal) and economy (partition count smaller than set size).

This homomorphism $\varphi: S_4 \to S_3$ tracks how permutations of four elements permute the three partitions. Its kernel contains precisely those permutations preserving all partitions: the Klein four-group $V_4 = \{e, (12)(34), (13)(24), (14)(23)\}$. This renders $A_4$ non-simple; indeed $A_n$ is simple for $n\ge 5$, making $n=4$ the only non-simple case among $n\ge 3$.

This manifests in Lie theory through the decomposition $\mathrm{SO}(4) \cong (\mathrm{SU}(2) \times \mathrm{SU}(2))/\mathbb{Z}_2$, splitting the Lie algebra $\mathfrak{so}(4) \cong \mathfrak{so}(3) \oplus \mathfrak{so}(3)$. This corresponds to decomposing 2-forms into self-dual and anti-self-dual components: $\Lambda^2(\mathbb{R}^4) = \Lambda^2_+ \oplus\Lambda^2_-$, where the Hodge star operator satisfies $\star^2 = 1$ on oriented Riemannian 4-manifolds (and $\star^2 = -1$ on 2-forms in Lorentzian signature), yielding eigenspaces with eigenvalues $\pm 1$ in the Euclidean case.

In gauge theory, this splitting transforms second-order Yang–Mills equations into first-order conditions. A connection with curvature $F$ satisfying $F = \star F$ (self-dual) or $F = -\star F$ (anti-self-dual) automatically solves $D\star F = 0$ since the Bianchi identity guarantees $DF = 0$. This dimensional coincidence — that the electromagnetic field strength is a 2-form and its Hodge dual has the same rank — makes Maxwell's equations acquire their most natural geometric expression in four dimensions. Here the equation $F=\pm \star F$ for 2-forms is specific to four dimensions; higher-dimensional analogues (e.g., $G_2$-instantons and $\mathrm{Spin}(7)$-instantons) exist but differ in structure.

Donaldson's theorem, a hallmark of four-dimensionality, uses this. The moduli space of anti-self-dual connections on a 4-manifold yields polynomial invariants distinguishing smooth structures. Two homeomorphic 4-manifolds may have different Donaldson invariants, proving they are not diffeomorphic — a phenomenon occurring in no other dimension. The identity $4 = 2 + 2$ enables the entire apparatus.

So the identity $4 = 2 + 2$ creates the alternating group exception, the Lie algebra splitting, the self-duality decomposition, and the instanton solutions that distinguish four-dimensional gauge theory. Stable orbits, renormalizable interactions, exotic smooth structures — these may indeed stem from this single combinatorial fact. The most sophisticated features of our universe follow from the simplest patterns in the integers.

For more details on the underlying mathematics, I recommend the book \href{https://bookstore.ams.org/FOURMAN}{The Wild World of 4-Manifolds} by Alexandru Scorpan.

\begin{commentary}[Why Four Might Be "Special"]
The convergence of independent constraints — orbit stability, renormalizability, Huygens principle, division algebras, gauge theory, the arithmetic identity $4=2+2$ — all pointing to four dimensions invites three interpretations. First, four may encode a fundamental truth of geometry, where mathematical coherence uniquely selects this dimensionality as the only one supporting complex, predictable structures. Second, the apparent necessity may reflect selection bias: we observe four dimensions because observers can only arise where physics permits stable atoms and chemistry, rendering our conclusion inevitable yet uninformative about whether other dimensionalities "exist" in some broader sense. Third, the entire exercise may be backfitting logic to a random parameter — finding post-hoc explanations for an arbitrary feature of our universe, mistaking coincidence for profundity. 

The proliferation of independent mathematical arguments favoring four suggests the first interpretation, yet the arguments themselves presuppose frameworks (differential geometry, quantum field theory, group representation theory) constructed within and calibrated to a four-dimensional universe. Whether these constraints reveal something deeper or merely echo the assumptions embedded in our theories remains unresolved. 
\end{commentary}

