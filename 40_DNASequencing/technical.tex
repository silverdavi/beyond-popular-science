\begin{technical}
{\Large\textbf{Illumina Sequencing and De Bruijn Assembly}}\\[0.3em]

\techheader{Sequencing by Synthesis}\\[0.5em]
Illumina uses reversible terminators with cleavable fluorescent labels. Each cycle:
\begin{align*}
&\text{DNA}_n + \text{dNTP-3'-block-fluor}\\
&\quad\xrightarrow{\text{pol}} \text{DNA}_{n+1}\\
&\text{Imaging} \rightarrow \text{Base identification}\\
&\text{Chemical cleavage} \rightarrow \text{3'-OH restoration}
\end{align*}

Bridge amplification creates clonal clusters ($\sim10^3$ copies) on flow cell surface. Fluorescence signal $S \propto N_{\text{mol}}$ enables base calling with error rate $\varepsilon \approx 0.1\%$.

\techheader{De Bruijn Graph Construction}\\[0.5em]
For read set $\mathcal{R}$ with k-mer length $k$:
\begin{align*}
V &= \{w \in \Sigma^{k-1} : w \text{ is prefix or suffix}\\
&\quad\text{of some k-mer in } \mathcal{R}\}\\
E &= \{w \in \Sigma^k : w \text{ appears in } \mathcal{R}\}
\end{align*}
where each k-mer edge connects its $(k-1)$-mer prefix to its $(k-1)$-mer suffix.

Each read of length $L$ contributes $L-k+1$ k-mer edges, compressing redundant sequence information into a compact graph structure.

\techheader{Eulerian Path Assembly}\\[0.5em]
Assembly seeks Eulerian path through $G$:
\begin{align*}
\text{Path} &= e_1e_2...e_m \text{ where}\\
&\quad \forall i: \text{tail}(e_i) = \text{head}(e_{i+1})\\
\text{Genome} &= e_1[1..k] + e_2[k] + e_3[k]\\
&\quad + ... + e_m[k]
\end{align*}
where $e_i[k]$ denotes the last base of k-mer $e_i$.

For an Eulerian path to exist, the underlying graph over nonzero-degree vertices must be weakly connected, with all vertices balanced (in-degree = out-degree), or exactly two semi-balanced vertices: one with out-degree = in-degree + 1 and one with in-degree = out-degree + 1.

\techheader{Coverage and k-mer Selection}\\[0.5em]
Expected k-mer coverage:
\[C_k = C_{\text{read}} \cdot \frac{L-k+1}{L}\]
where $C_{\text{read}} = NL/G$ (reads $\times$ length / genome).

Optimal $k$ balances a tradeoff: smaller $k$ yields more connections and higher coverage but introduces ambiguity, while larger $k$ reduces repeat ambiguity at the cost of lower coverage and potential gaps.

Typically $k \in [50, 250]$ for Illumina data.

\techheader{Graph Complexity}\\[0.5em]
Real graphs contain three main types of structural features: \textbf{bubbles} (parallel paths created by SNPs or errors), \textbf{tips} (dead ends from coverage gaps), and \textbf{repeats} (creating branching and convergence). Error correction typically removes k-mers with coverage below a threshold.

\techheader{Paired-End Constraints}\\[0.5em]
Insert size $d \sim \mathcal{N}(\mu, \sigma^2)$ provides scaffolding:
\[|p(r_1, r_2) - \mu| < 3\sigma\]
where $p(r_1, r_2)$ is genomic distance between read pairs.

\techref
{\footnotesize
Bentley et al. (2008). \textit{Nature} 456:53-59.\\
Pevzner et al. (2001). \textit{PNAS} 98:9748-9753.
}
\end{technical}