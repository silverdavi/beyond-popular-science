This project began as a small “flight magazine” I put together for my family before a long transatlantic trip. I started filling it with notes I had collected over the years — questions that seem simple on the surface but are, on inspection, scientifically intricate and mathematically deep.

After drafting a few pieces — why gold is yellow (relativity) and why apples fall from trees (time dilation) — it became clear this was more than a pamphlet. Throughout the day, I found myself recalling phenomena I had once wondered about, and the list of topics kept growing.

Anyone who knows me knows that every so often I get excited about some new piece of science and insist on explaining it, regardless of the time or place. It turned out there were many such topics. One thing led to another; more pages were written, and this became a real book.

I owe special thanks to everyone who supported the early stages of this project — both personally and through the Kickstarter campaign — and helped make the first printing possible. I am especially grateful to my wife, Enny, for her unwavering support, patience, and love. I am also thankful to my kids, who patiently sat through countless explanations of physics and mathematics.

This book returns to the roots of scientific wonder, combining accessible explanations with rigorous mathematical foundations. Unlike contemporary science communication that oversimplifies or sensationalizes, it highlights the beauty of science as it truly is: both elegant and complex. The focus is understanding, not just exposure.

Too often, modern science communicators rely on a “laugh track” approach — telling readers how they should feel (“This is mind-blowing!”) instead of letting wonder arise naturally from the ideas. This cheapens the experience, as though science requires manufactured excitement. Science doesn't need exaggeration; its wonder is self-evident to those who explore it properly.

I must apologize if my enthusiasm and flair are not easy to convey in this medium. But I assure you: even reading portions of this book should leave you feeling that our universe is more fantastical than any Tolkien creation. The effects we observe in the natural world work in wondrous ways — relativity and quantum mechanics are stranger than fiction, with more sorcerous underlying complexity than any mythological chant.

When a ray of sunlight hits your eyes and you move, the cascade of events is magnificent: trillions of quantum field excitations coordinate changes across millions of city-scale structures. Cells with politics and defense protocols — ribosomes pounding out translations like factories, mitochondria running proton gradients as power plants, immune patrols scanning for invaders, membranes running checkpoints and visa systems — trillions of these cities operating in parallel just to send an impulse down the spinal cord.

Every molecule performs Hamiltonian plays, issuing redistribution orders to orbitals. Atoms are not little spheres but dense arrangements of nuclei with surrounding clouds, and the protons and neutrons in those nuclei are bound states of three colorful quarks, constantly borrowing energy from the vacuum, exchanging gluons trillions of times per second, stitching color fields so tight that the binding energy exceeds the sum of the parts, generating most of the mass that weighs the body down, mass that resists acceleration, mass that makes clocks tick more slowly.

Layered above, enzymes catalyze reactions in femtoseconds, metabolic pathways route energy into ATP, blood pumps uphill against gravity, oxygen convoys carried by hemoglobin through capillary labyrinths. Neurons fire spikes, action potentials race along axons, vesicles dump neurotransmitters into synapses, inhibitory and excitatory votes cast trillions of times per second, motor cortex computes commands, motoneurons release acetylcholine into neuromuscular junctions, muscle fibers flood with calcium, actin and myosin filaments slide, sarcomeres shorten, tendons tug, bones shift, and the person moves.

And the photon itself was generated in a star's core where the weak nuclear force converts protons to neutrons after quantum tunneling through an energy barrier, then trapped in plasma for a million years in random-walk collisions, finally escaping the surface and flying straight for minutes across the vacuum (zero time from the photon's point of view), striking your retina and flipping rhodopsin from cis to trans — a femtosecond molecular rearrangement amplified into a millisecond spike.

The cascade from subnuclear quark fields to stellar photon journeys to cellular cities to muscular contraction all chained together so that when you think “I should move,” your body shifts in space and every layer of physics and biology has fired in unison to make it happen.

This must be less mundane than any grumpy villain that can fly forks around telekinetically. Maybe after reading a few chapters you will agree with this claim even more.

All topics in this book have personal stories behind them — I remember how I learned about them. \textcolor{red!80!black}{I hope I can infect you with some of that excitement.}

The goal is to respect the reader's intelligence and curiosity. Whether discussing topological insulators, the mechanics of atomic clocks, or the subtleties of time dilation, these chapters present science as it is: demanding, rewarding, and truly inspiring.

This book counteracts oversimplified science communication. Science isn't slogans or easy answers — its complexity is a feature to celebrate. Understanding takes effort, but transforms fleeting curiosity into lasting enlightenment.

If you’re ready to explore science in its full intellectual glory, I invite you to turn the page.

