\begin{technical}
{\Large\textbf{Calendar Mathematics}}\\[0.2em]

\techheader{Molad Calculation}\\[0.3em]
The traditional Jewish lunar month length:
\[
29d\,12h\,44m\,3\tfrac{1}{3}s = 29.530594\text{ days}
\]
Modern astronomical value: 29.530589 days. Error accumulates at $\Delta = 5 \times 10^{-6} \times N$ \text{days} where $N$ is months elapsed. After 1000 years ($\approx$12,400 months): error $\approx$ 1.5 hours.

\techheader{Metonic Cycle}\\[0.3em]
19 solar years $\approx$ 235 lunar months:
\[
19 \times 365.2422 = 6939.602\text{ days}
\]
\[
235 \times 29.530594 = 6939.689\text{ days}
\]
Difference: 0.087 days per 19-year cycle. Leap years occur in years 3, 6, 8, 11, 14, 17, 19.

Mathematically, the pattern of leap years in this 19-year cycle has almost the same “tempo” as the diatonic major scale, a connection pointed out to me by Amit B. If we look at the gaps between leap years we obtain the circular sequence
\[
(3,2,3,3,3,2,3),
\]
consisting of five “long” gaps of 3 years and two “short” gaps of 2 years, distributed as evenly as possible around the 19-year cycle.  The major scale does exactly the same thing on a 12-step chromatic circle: it uses seven notes arranged with five whole-tone steps and two semitone steps in the maximally even pattern
\[
(2,1,2,2,2,1,2).
\]
In both cases we are placing \(k=7\) marked points on a cycle of length \(N\) with \(N \equiv 5 \pmod 7\) (here \(N=19\) years or \(N=12\) semitones).  The average step sizes
\[
12/7 = 1 + 5/7, \qquad 19/7 = 2 + 5/7
\]
share the fractional part $5/7$, which forces the “five long, two short” pattern and fixes their relative ordering.  Abstractly, the leap-year cycle and the major scale are two realizations of the same maximally even 7-beat rhythm.

\techheader{Dechiyot (Postponements)}\\[0.2em]
Rosh Hashanah cannot fall on Sun, Wed, or Fri:
\begin{enumerate}[leftmargin=*,topsep=0pt,itemsep=0pt]
\item \textbf{Lo ADU}: Direct postponement
\item \textbf{Molad Zaken}: If molad $\geq$ 18:00
\item \textbf{GaTRaD}: Regular year, Tuesday $\geq$ 9h 204p
\item \textbf{BeTuTaKPaT}: After leap, Monday $\geq$ 15h 589p
\end{enumerate}

\techheader{Ben Meir Dispute (922 CE)}\\[0.2em]
Ben Meir: Molad threshold = 642 parts\\
Traditional: Molad threshold = 0 parts\\
For Tishrei 4683 (922 CE):\\
Ben Meir: Day 2, 9h 204p\\
Saadia: Day 2, 15h 589p\\

\techheader{Polar Day Solutions}\\[0.2em]
\textit{Sun-circle method}: Each 24h circuit = 1 day\\
\textit{Origin timezone}: Follow departure location\\
\textit{Proportional}: Calculate theoretical solar angle: $h = 15°(t-12) - \lambda + E$

\techheader{Classical Source}\\[0.2em]
\textit{Halakhot Pesuqot} (Rav Yehudai Gaon, 8th century):

\begin{hebrew}
לעולם ראש חדש אדר סמוך לניסן הוא ערב הפסח,
והפסח הוא ערב העצרת,
והעצרת הוא ערב ראש השנה.
לא בד״ו פסח,
לא גה״ז עצרת,
לא אד״ו ראש השנה וסוכה,
לא אג״ו יום הכיפורים,
ולא זבד פורים.
\end{hebrew}

\textit{Translation:}\\
Always, the new moon of Adar close to Nisan is the eve of Passover; Passover precedes Shavuot; and Shavuot precedes Rosh Hashanah. Passover never occurs on days \texthebrew{בד״ו} (MoWeFr); Shavuot never on \texthebrew{גה״ז} (TuThSa); Rosh Hashanah and Sukkot never on \texthebrew{אד״ו} (SuWeFr); Yom Kippur never on \texthebrew{אג״ו} (SuTuFr); and Purim never on \texthebrew{זבד} (MoWeSa).

\techref
{\footnotesize
Feldman, W.M. (1931). \textit{Rabbinical Mathematics and Chronology}.\\
Stern, S. (2019). \textit{The Jewish Calendar Controversy of 921/2}.\\
Lipschitz, Y. (1850). \textit{Tiferet Yisrael}, Berakhot 1.
}
\end{technical}